%!TEX root = appunti.tex
%!TEX encoding = UTF-8 Unicode
\chapter{Cenni di calcolo combinatorio}
Il calcolo combinatorio è una branca della matematica che studia l'enumerazione, la combinazione e la permutazione di insiemi di elementi e le relazioni matematiche che caratterizzano queste proprietà.
\section{Disposizioni} % (fold)
\begin{definition}[Disposizione con ripetizione]
  Il numero di modi di prendere $k$ elementi \textbf{ordinati} da $n$ possibili \textbf{con ripetizione}.
  \[ n \cdot n \cdots n = n^k \]
\end{definition}
Perché il primo elemento può essere scelto tra $n$ elementi, il secondo anche, e così via fino al $k$-esimo elemento.

\begin{definition}[Disposizione semplice]\label{def:disposizione_semplice}
  Il numero di modi di prendere $k$ elementi \textbf{ordinati} da $n$ possibili \textbf{senza ripetizione}.
  \[ n \cdot (n - 1) \cdots (n - k + 1) = \frac{n!}{(n-k!)} \]
\end{definition}
Perché il primo elemento può essere scelto tra $n$ elementi, il secondo tra $n - 1$ elementi, il terzo tra $n - 2$ e così via fino all'$k$-esimo elemento.

\begin{definition}[Permutazione]
  È una disposizione semplice con $k = n$.
  \[ \frac{n!}{(n - n)!} = n! \]
\end{definition}

% section disposizioni (end)

\section{Combinazioni} % (fold)
\begin{definition}[Combinazione semplice]
  Il numero di modi di prendere $k$ elementi \textbf{non ordinati} da $n$ possibili \textbf{senza ripetizione}. Conosciuto anche come coefficiente binomiale.
  \[ \binom{n}{k} = \frac{n!}{k!(n-k)!} \]
\end{definition}
% TODO spiegare perché si usa la combinazione semplice

\begin{definition}[Combinazione con ripetizione]
  Il numero di modi di prendere $k$ elementi \textbf{non ordinati} da $n$ possibili \textbf{con ripetizione}.
  \[ \binom{n+k-1}{k} = (n - 1, k)! \]
  dove \( (n - 1, k)! \) è un coefficiente multinomiale\footnote{vedi definizione \ref{def:coefficiente_multinomiale}}.
\end{definition}
% TODO spiegare perché si usa la combinazione con ripetizione

% section combinazioni (end)

\section{Coefficiente multinomiale} % (fold)
\begin{definition}\label{def:coefficiente_multinomiale}
  Il numero di modi di formare $k$ gruppi di \( n_1, \ldots, n_k \) elementi ciascuno.
  \[ (n_1, n_2, \ldots, n_k)! = \frac{(n_1 + n_2 + \ldots + n_k)!}{n_1! \cdot n_2! \cdots n_k!} \]
\end{definition}
% TODO spiegare perché si usa il coefficiente multinomiale

% section coefficiente_multinomiale (end)
