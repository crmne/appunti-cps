%!TEX root = appunti.tex
%!TEX encoding = UTF-8 Unicode
\chapter{Distribuzioni discrete}
\begin{definition}[Distribuzione discreta]\label{def:distribuzione_discreta}
  Una distribuzione statistica le cui variabili possono avere solo valori discreti.

  Dunque, un numero aleatorio $X$ si dice con distribuzione discreta se la cardinalità di $I(X)$ è finita o numerabile.
\end{definition}
\begin{definition}[Distribuzione di probabilità]\label{def:distribuzione_di_probabilita}
  La funzione che descrive la probabilità che un certo valore si verifichi.
  \marginnote{La distribuzione di probabilità viene anche detta ``funzione di densità di probabilità''.}

  Se consideriamo $X$ una variabile aleatoria con distribuzione discreta, la sua distribuzione di probabilità sarà data da
  \[ \pr(X = x_i) = p(x_i) \text{ con } x_i \in I(X) \]
  dove
  \[ \sum_{x_i \in I(X)} \pr(X = x_i) = 1 \]
\end{definition}

\begin{definition}[Schema di Bernoulli]\label{def:schema_bernoulli}
  Una successione \( (E_i)_{i \in \mathbb{N}} \) di eventi stocasticamente indipendenti ed equiprobabili, ovvero tali per cui vale che
  \[ \pr(E_i) = p, ~ \forall i \in \mathbb{N} \]
\end{definition}
