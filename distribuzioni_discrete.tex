%!TEX root = appunti.tex
%!TEX encoding = UTF-8 Unicode
\chapter{Distribuzioni discrete}
\begin{definition}[Distribuzione discreta]\label{def:distribuzione_discreta}
  Una distribuzione statistica le cui variabili possono avere solo valori discreti.

  Dunque, un numero aleatorio $X$ si dice con distribuzione discreta se la cardinalità di $I(X)$ è finita o numerabile.
\end{definition}
\begin{definition}[Distribuzione di probabilità]\label{def:distribuzione_di_probabilita}
  La funzione che descrive la probabilità che un certo valore si verifichi.
  \marginnote{La distribuzione di probabilità viene anche detta ``funzione di densità di probabilità''.}

  Se consideriamo $X$ una variabile aleatoria con distribuzione discreta, la sua distribuzione di probabilità sarà data da
  \[ \pr(X = x_i) = p(x_i) \text{ con } x_i \in I(X) \]
  dove
  \[ \sum_{x_i \in I(X)} \pr(X = x_i) = 1 \]
\end{definition}

\begin{definition}[Schema di Bernoulli]\label{def:schema_bernoulli}
  Una successione \( (E_i)_{i \in \mathbb{N}} \) di eventi stocasticamente indipendenti ed equiprobabili, ovvero tali per cui vale che
  \[ \pr(E_i) = p, ~ \forall i \in \mathbb{N} \]
\end{definition}

\section{Distribuzione binomiale}
\begin{figure*}
  \includegraphics{distribuzione_binomiale}
  \caption{Distribuzione binomiale % e la sua approssimazione a distribuzione normale (vedi definizione \ref{def:distribuzione_normale})
  } 
\end{figure*}
\begin{definition}[Distribuzione binomiale]\label{def:}
  La distribuzione discreta di probabilità
  \footnote{vedi definizione \ref{def:distribuzione_di_probabilita}}
  di ottenere $k$ successi su $n$ prove di Bernoulli
  \footnote{ovvero dove il risultato è positivo con probabilità $p$ e negativo con probabilità \( (1-p) \), vedi definizione \ref{def:schema_bernoulli}}.  
\end{definition}

Dato uno Schema di Bernoulli \( (E_i)_{i \in \mathbb{N}} \) definiamo il numero di successi su $n$ prove come
\[ S_n = (E_1 + \ldots + E_n) \]
dove, ovviamente
\[ I(S_n) = \{0,1, \ldots, n\} \]

La distribuzione di probabilità dunque è, tramite i costituenti:
\[ \pr(S_n = k) = \sum_{Q \subset (S_n = k)} \pr(Q) \]
ovvero dobbiamo sommare tutte le probabilità dei costituenti del primo tipo dell'evento \( (S_n = k) \), come, ad esempio
\[ Q = E_1 \cdots E_k \tilde{E_{k+1}} \cdots \tilde{E_n}\]
che rappresenta l'evento in cui i $k$ successi si sono ottenuti con le prime $k$ prove.
Analogamente ogni altro costituente di \( (S_n = k) \) conterrà $k$ eventi che si sono verificati e \( n-k \) che non si sono verificati.
Siccome tutti gli eventi sono indipendenti ed equamente distribuiti, ogni costituente $Q$ ha la stessa probabilità, pari a
\[ \pr(Q) = p^k (1-p)^{n-k} \]

Dunque per avere \( \pr(S_n = k) \), basta contare i costituenti.
Essi sono uguali al numero di modi di scegliere i $k$ eventi che si verificano sugli $n$ totali. Si ottiene quindi
\[
  \pr(S_n = k) = \binom{n}{k} p^k (1-p)^{n-k} 
\]

Si dice dunque che \( S_n \) ha distribuzione binomiale \( \mathcal{B}(n, p) \) dove $n$ è il numero degli eventi e $p$ la probabilità di ognuno.
