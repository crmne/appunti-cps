%!TEX TS-program = pdflatex
%!TEX encoding = UTF-8 Unicode
%
%  Appunti di Calcolo delle Probabilità e Statistica
%
%  Copyright (c) 2010 Carmine Paolino. Some rights reserved.
%
%  This work is licensed under the Creative Commons
%  Attribution-Noncommercial-Share Alike 2.5 Italy License. To view a copy of
%  this license, visit http://creativecommons.org/licenses/by-nc-sa/2.5/it/ or
%  send a letter to Creative Commons, 171 Second Street, Suite 300,
%  San Francisco, California, 94105, USA.
%
\documentclass[a4paper]{tufte-book}

% Colorize links
\hypersetup{colorlinks}

% For nicely typeset tabular material
\usepackage{booktabs}

% Use italian as document language
\usepackage[italian]{babel}

% More symbols
\usepackage{amsmath}
\usepackage{amssymb}
\usepackage{amsthm}

% Check if you are using PDFLaTeX:
\usepackage{ifpdf}

\ifpdf
\usepackage[pdftex]{graphicx}
\DeclareGraphicsExtensions{.pdf, .jpg, .tif}
\usepackage[utf8]{inputenc}
\else
\usepackage{graphicx}
\DeclareGraphicsExtensions{.eps, .jpg}
\fi

% Check if you are using XeLaTeX:
\usepackage{ifxetex}

\ifxetex
\usepackage{xunicode}
\usepackage{xltxtra}
\fi

% Math stuff
\newtheorem{proposition}{Proposizione}
\newtheorem{theorem}{Teorema}
\newtheorem{corollary}[theorem]{Corollario}
% \theoremstyle{definition}
\newtheorem{definition}{Definizione}
\theoremstyle{remark}
\newtheorem*{example}{Esempio}
\newcommand{\cov}{\mathbf{cov}}
\newcommand{\sqm}{\boldsymbol\sigma}
\newcommand{\var}{\sqm^2}
\newcommand{\pr}{\mathbf{P}}
\newcommand{\cc}{\boldsymbol\rho}
\DeclareMathOperator{\sgn}{sgn}

% Content starts here
\title{Appunti di Calcolo delle Probabilità e Statistica}
\author{Carmine Paolino}

\begin{document}
\frontmatter
\maketitle

%!TEX root = appunti.tex
%!TEX encoding = UTF-8 Unicode
%
%  Appunti di Calcolo delle Probabilità e Statistica
%
%  Copyright (c) 2010 Carmine Paolino. Some rights reserved.
%
%  This work is licensed under the Creative Commons
%  Attribution-Noncommercial-Share Alike 2.5 Italy License. To view a copy of
%  this license, visit http://creativecommons.org/licenses/by-nc-sa/2.5/it/ or
%  send a letter to Creative Commons, 171 Second Street, Suite 300,
%  San Francisco, California, 94105, USA.
%
\newpage
\begin{fullwidth}
~\vfill
\thispagestyle{empty}
\setlength{\parindent}{0pt}
\setlength{\parskip}{\baselineskip}
Copyright \copyright\ \the\year\ \thanklessauthor

\par\smallcaps{github.com/earcar/appunti-cps}

\par This work is licensed under the Creative Commons Attribution-Noncommercial-Share Alike 2.5 Italy License. To view a copy of this license, visit \url{http://creativecommons.org/licenses/by-nc-sa/2.5/it/} or send a letter to Creative Commons, 171 Second Street, Suite 300, San Francisco, California, 94105, USA.\index{license}

\end{fullwidth}

\tableofcontents

\mainmatter
% TODO formula di Stirling
%!TEX root = appunti.tex
%!TEX encoding = UTF-8 Unicode
%
%  Appunti di Calcolo delle Probabilità e Statistica
%
%  Copyright (c) 2010 Carmine Paolino. Some rights reserved.
%
%  This work is licensed under the Creative Commons
%  Attribution-Noncommercial-Share Alike 2.5 Italy License. To view a copy of
%  this license, visit http://creativecommons.org/licenses/by-nc-sa/2.5/it/ or
%  send a letter to Creative Commons, 171 Second Street, Suite 300,
%  San Francisco, California, 94105, USA.
%
\chapter{Cenni di calcolo combinatorio}
Il calcolo combinatorio è una branca della matematica che studia l'enumerazione, la combinazione e la permutazione di insiemi di elementi e le relazioni matematiche che caratterizzano queste proprietà.
\section{Disposizioni}
\begin{definition}[Disposizione con ripetizione]
  Il numero di modi di prendere $k$ elementi \textbf{ordinati} da $n$ possibili \textbf{con ripetizione}.
  \[
  n \cdot n \cdots n = n^k
  \]
\end{definition}
Perché il primo elemento può essere scelto tra $n$ elementi, il secondo anche, e così via fino al $k$-esimo elemento.

\begin{definition}[Disposizione semplice]\label{def:disposizione_semplice}
  Il numero di modi di prendere $k$ elementi \textbf{ordinati} da $n$ possibili \textbf{senza ripetizione}.
  \[
  n \cdot (n - 1) \cdots (n - k + 1) = \frac{n!}{(n-k!)}
  \]
\end{definition}
Perché il primo elemento può essere scelto tra $n$ elementi, il secondo tra $n - 1$ elementi, il terzo tra $n - 2$ e così via fino all'$k$-esimo elemento.

\begin{definition}[Permutazione]
  È una disposizione semplice con $k = n$.
  \[
  \frac{n!}{(n - n)!} = n!
  \]
\end{definition}

\section{Combinazioni}
\begin{definition}[Combinazione semplice]
  Il numero di modi di prendere $k$ elementi \textbf{non ordinati} da $n$ possibili \textbf{senza ripetizione}. Conosciuto anche come coefficiente binomiale.
  \[
  \binom{n}{k} = \frac{n!}{k!(n-k)!}
  \]
\end{definition}
% TODO spiegare perché si usa la combinazione semplice

\begin{definition}[Combinazione con ripetizione]
  Il numero di modi di prendere $k$ elementi \textbf{non ordinati} da $n$ possibili \textbf{con ripetizione}.
  \[
  \binom{n+k-1}{k} = (n - 1, k)!
  \]
  dove \( (n - 1, k)! \) è un coefficiente multinomiale\footnote{vedi definizione \ref{def:coefficiente_multinomiale}}. 
\end{definition}
% TODO spiegare perché si usa la combinazione con ripetizione

\section{Coefficiente multinomiale}
\begin{definition}\label{def:coefficiente_multinomiale}
  Il numero di modi di formare $k$ gruppi di \( n_1, \ldots, n_k \) elementi ciascuno.
  \[
  (n_1, n_2, \ldots, n_k)! = \frac{(n_1 + n_2 + \ldots + n_k)!}{n_1! \cdot n_2! \cdots n_k!}
  \]
\end{definition}
% TODO spiegare perché si usa il coefficiente multinomiale

%!TEX root = appunti.tex
%!TEX encoding = UTF-8 Unicode
%
%  Appunti di Calcolo delle Probabilità e Statistica
%
%  Copyright (c) 2010 Carmine Paolino. Some rights reserved.
%
%  This work is licensed under the Creative Commons
%  Attribution-Noncommercial-Share Alike 2.5 Italy License. To view a copy of
%  this license, visit http://creativecommons.org/licenses/by-nc-sa/2.5/it/ or
%  send a letter to Creative Commons, 171 Second Street, Suite 300,
%  San Francisco, California, 94105, USA.
%
\chapter{Numeri aleatori}
\section{Introduzione}
La probabilità è la branca della matematica che studia i possibili esiti di eventi dati insieme alle relative verosimiglianze e distribuzioni.

Nell'uso comune, il termine ``probabilità'' è usato per indicare la possibilità che un evento particolare (o un insieme di eventi) si verificherà, espressa su una scala lineare da 0 (impossibilità) a 1 (certezza) o come una percentuale compresa tra 0 e 100\%.
L'analisi degli eventi disciplinati dalla probabilità si chiama statistica.
% TODO accennare a De Finetti

\newthought{Gli oggetti fondamentali} della probabilità sono i numeri aleatori\footnote{detti anche variabili casuali, variabili aleatorie o variabili stocastiche}.

\begin{definition}[Numero aleatorio]
  Un numero ben determinato ma non noto per mancanza di informazioni.
\end{definition}
È pratica comune denotare i numeri aleatori con le lettere maiuscole (più comunemente $X$).
L'\emph{insieme dei valori possibili}\footnote{detto anche spazio campionario} che $X$ può assumere viene denotato con $I(X)$.

Caso particolare dei numeri aleatori sono i \emph{numeri certi}, quantità conosciute dove $I(X)$ contiene un solo elemento.

\begin{definition}[Continuità dei numeri aleatori]
  Un numero aleatorio $X$ si dice
  \begin{itemize}
  \item \emph{limitato superiormente} se \( \sup I(X) < + \infty \)
  \item \emph{limitato inferiormente} se \( \inf I(X) > - \infty \)
  \item \emph{limitato} se limitato superiormente e inferiormente
  \end{itemize}
\end{definition}

\begin{definition}[Indipendenza logica tra numeri aleatori]
  I numeri aleatori $X_1, \ldots, X_n$ si dicono logicamente indipendenti se
  \marginnote{Il simbolo $\times$ denota il prodotto scalare.}
  \[ I(X_1, \ldots, X_n) = I(X_1) \times \ldots \times I(X_n) \]
\end{definition}

\begin{definition}[Operazioni sui numeri aleatori]
  Oltre alle canoniche operazioni aritmetiche, i numeri aleatori presentano le seguenti operazioni:

  Somma logica:
  \[ X \vee Y \equiv \max(X,Y) \]

  Prodotto logico:
  \[ X \wedge Y \equiv \min(X,Y) \]

  Complementazione:
  \[ \tilde{X} \equiv 1 - X \]
\end{definition}

\begin{definition}[Proprietà delle operazioni sui numeri aleatori]{\ \\} % trick per aggiungere un a-capo in più
  Distributiva:
  \begin{align*}
    X \vee (Y \wedge Z) = (X \vee Y) \wedge (X \vee Z) \\
    X \wedge (Y \vee Z) = (X \wedge Y) \vee (X \wedge Z)
  \end{align*}

  Associativa:
  \begin{align*}
    X \vee (Y \vee Z) = (X \vee Y) \vee Z \\
    X \wedge (Y \wedge Z) = (X \wedge Y) \wedge Z
  \end{align*}

  Commutativa:
  \begin{align*}
    X \vee Y = Y \vee X \\
    X \wedge Y = Y \wedge X
  \end{align*}

  Involutiva:
  \[ \tilde{\tilde{X}} = X \]

  De Morgan:
  \begin{align*}
    (X \vee Y) \; \tilde{} = \tilde{X} \wedge \tilde{Y} \\
    (X \wedge Y) \; \tilde{} = \tilde{X} \vee \tilde{Y}
  \end{align*}
\end{definition}

\section{Eventi}
\begin{definition}[Evento]\label{def:evento}
  Caso particolare di numero aleatorio, dove \( I(E) \subseteq {0,1} \).
\end{definition}
Per dichiarare un evento si usa la seguente sintassi:
\[ E = (\text{condizione}) \]
che indica che se la condizione si verifica allora \( E = 1 \), altrimenti \( E = 0 \).

\begin{definition}[Asserzione]\label{def:asserzione}
  Se $E$ è una proposizione o un evento allora, usando il simbolo $\vdash$ come prefisso, \( \vdash E \) diventa l'asserzione che ``$E$ è certo''. 
\end{definition}

\begin{definition}[Implicazione]\label{def:implicazione}
  Un evento $A$ implica un evento $B$ quando $A$ non si può verificare se non si verifica anche $B$, o che \( A\tilde{B} \) è impossibile.

  Denotiamo dunque ``$A$ implica $B$'' con \( A \subseteq B \), che ``$A$ è identico a $B$'' con \( A \equiv B \) e che ``$A$ implica strettamente $B$'' con \( A \subset B \).
\end{definition}

\begin{proposition}[Operazioni su eventi]{\ \\}
  Somma logica:
  \marginnote{La somma logica di eventi può essere vista come un OR booleano.}
  \[ E \vee F = E + F - EF \]

  Prodotto logico:
  \marginnote{Analogamente il prodotto logico può essere visto come un AND booleano.}
  \[ E \wedge F = EF \]
\end{proposition}
\begin{proof}
  Somma logica:
  \begin{equation*}
    \begin{split}
      E \vee F &= (E \vee F) ~ \tilde{} ~ \tilde{} \\
      &= (\tilde{E} \wedge \tilde{F}) ~ \tilde{} \\
      &= ((1-E)(1-F)) ~ \tilde{} \\
      &= (1 - E - F + EF) ~ \tilde{} \\
      &= 1 - 1 + E + F - EF \\
      &= E + F - EF
    \end{split}
  \end{equation*}

  Prodotto logico:
  \begin{equation*}
    \begin{split}
      E \vee F \vee G &= (E \vee F \vee G) ~ \tilde{} ~ \tilde{} \\
      &= ((1 - E)(1 - F)(1 - G)) ~ \tilde{} \\
      &= (1 - F - E - G + EF + FG + EG - EFG) ~ \tilde{} \\
      &= 1 - 1 + F + E + G - EF - FG - EG + EFG \\
      &= F + E + G - EF - FG - EG + EFG
    \end{split}
  \end{equation*}
\end{proof}

\begin{definition}[Proprietà degli eventi]
  Per \( E_1, \ldots, E_n \) eventi

  Incompatibilità:
  \( \vdash E \cdot F = 0 \)

  Mutua esclusione:
  \( \forall i, j \text{ con } i \neq j ~ \vdash E_i \cdot E_j = 0 \)
  \marginnote{e dunque \( \vdash E_1 + \ldots + E_n \leq 1 \)}

  Esaustività:
  \( \vdash E_1 + \ldots + E_n \geq 1 \)

  Partizione
  \footnote{ovvero mutuamente esclusivi ed esaustivi}:
  \( \vdash E_1 + \ldots + E_n = 1 \)
\end{definition}

\subsection{Costituenti}
\marginnote{Intuitivamente un costituente è uno dei possibili modi di comporre gli eventi \( E_1, \ldots, E_n \).}
\begin{definition}[Costituente]
  Si definisce costituente degli eventi \( E_1, \ldots, E_n \) l'evento
  \[ C = E_1^* \cdots E_n^* \]
  dove
  \begin{equation*}
    E_i^* =
    \begin{cases}
      E_i \\
      \tilde{E_i}
    \end{cases}
  \end{equation*}
  Non tutti i costituenti sono possibili, tranne nel caso in cui gli $E_i$ sono logicamente indipendenti.
\end{definition}

Il numero dei costituenti possibili è sempre \( \leq 2^n \), eguaglia \( 2^n \) solo nel caso un cui gli $E_i$ sono logicamente indipendenti.

I costituenti possibili costituiscono una partizione:
\[ \sum_{C ~ \text{costituente}} C = (E_1 + \tilde{E_1}) \cdots (E_n + \tilde{E_n}) = 1\]

\begin{definition}[Tipologie di costituenti]
  Un costituente, rispetto ad un evento $E$, può essere di:
  \begin{itemize}
  \item \emph{I tipo} quando si verifica se $E$ si verifica: \( C \subset E \)
  \item \emph{II tipo} quando si verifica se $E$ non si verifica: \( C \subset \tilde{E} \)
  \item \emph{III tipo} altrimenti
  \end{itemize}
\end{definition}

\begin{definition}[Dipendenza logica tramite costituenti]
  Un particolare evento $E$ si dice:
  \begin{itemize}
  \item \emph{logicamente dipendente}
  \footnote{è sempre possibile determinarlo, dati i risultati degli eventi \( E_1, \ldots, E_n \)}
  dagli eventi \( E_1, \ldots, E_n \) se tutti i costituenti di \( E_1, \ldots, E_n \) sono del I o del II tipo
  \item \emph{logicamente indipendente}
  \footnote{il risultato di $E$ rimane incerto (possibile) anche quando il risultato di tutti gli altri, qualsiasi esso sia, è noto}
  dagli eventi \( E_1, \ldots, E_n \) se tutti i costituenti di \( E_1, \ldots, E_n \) sono del III tipo
  \item \emph{logicamente semidipendente}
  \footnote{il risultato di $E$ \emph{potrebbe} dipendere dai risultati degli altri eventi}
  dagli eventi \( E_1, \ldots, E_n \) altrimenti
  \end{itemize}
\end{definition}

Se $E$ è logicamente dipendente da \( E_1, \ldots, E_n \) si può scrivere
\[ E = \sum_{C \subset E} C \]

\section{Previsione e Probabilità}
\subsection{Previsione}
\marginnote{La previsione non tenta di indovinare il risultato, ma trasforma, molto ottimisticamente, un'incertezza in una certezza ben definita, ma di nessun valore oggettivo.}
\begin{definition}[Previsione]
  Esprime l'equivalente certo del fenomeno casuale rappresentato da un numero aleatorio $X$, e si denota con $\pr(X)$.

  La previsione non è oggettiva ma dipende dal nostro stato di informazione e dalla nostra valutazione.
\end{definition}

Non essendo oggettiva dobbiamo valutare e ragionare su ogni scelta e valutare i pro ed i contro secondo qualche criterio.

\begin{definition}[Criterio di coerenza]\label{def:coerenza}
  Ogni scelta va fatta in modo tale che le decisioni non portino a risultati non desiderati.
\end{definition}

\begin{example}[Scommessa]
  In una scommessa siamo chiamati a scegliere un certo valore \( \pr(X) = \bar{x} \) che ci dà un guadagno (aleatorio) proporzionale a \( X -\bar{x} \):
    \[ \lambda(X-\bar{x}) \]
  dove \( \lambda \) è un coefficiente di proporzionalità scelto da chi propone la scommessa.
  Dunque la scelta di $\bar{x}$, per il criterio di coerenza, non deve portare a perdite certe.
\end{example}

\newthought{Le seguenti proprietà} seguono dal criterio di coerenza:

\begin{proposition}[Monotonia della previsione]\label{pro:monotonia_previsione}
  \[ \inf I(X) \leq \pr(X) \leq \sup I(X) \]
\end{proposition}

\begin{proof}
  Supponendo che \( \bar{x} < \inf I(X) \), allora per \( \lambda < 0 \):
  \[ \vdash \lambda(X - \bar{x}) < 0 \]
  Perché $\lambda$ è una quantità negativa, e \( (X - \bar{x}) \) positiva.

  Se invece supponiamo \( \bar{x} > \sup I(X) \), allora per \( \lambda > 0 \):
  \[ \vdash \lambda(X - \bar{x}) < 0 \]
  Perché $\lambda$ è una quantità positiva, e \( (X - \bar{x}) \) negativa.

  Ne segue che per non avere perdite certe:
  \[ \inf I(X) \leq \pr(X) \leq \sup I(X) \]
\end{proof}

\begin{proposition}[Linearità della previsione]\label{pro:linearita_previsione}
  \[ X = \alpha_1 X_1 + \ldots + \alpha_n X_n \Rightarrow \alpha_1 \pr(X_1) + \ldots + \alpha_n \pr(X_n) \]
\end{proposition}

\begin{proof}
  Consideriamo il numero aleatorio \( Z = X + Y \) e posti \( \bar{z} = \pr(Z) \), \( \bar{x} = \pr(X) \) e \( \bar{y} = \pr(Y) \), sia \( G \) il guadagno
  \begin{align*}
    G &= c_1(X - \bar{x}) + c_2(Y - \bar{y}) + c_3(Z - \bar{z}) \\
    &= c_1(X - \bar{x}) + c_2(Y - \bar{y}) + c_3(X + Y - \bar{z}) \\
    &= X(c_1 + c_3) + Y(c_2 + c_3) - c_1\bar{x} - c_2\bar{y} - c_3\bar{z}
  \end{align*}

  Ponendo \( c_1 = c_2 = - c_3 \) annulliamo la parte aleatoria:
  \begin{align*}
    G &= c_3 \bar{x} + c_3 \bar{y} - c_3 \bar{z} \\
    &= c_3(\bar{x} + \bar{y} - \bar{z})
  \end{align*}

  Per evitare che si possa scegliere $c_3$ in modo che \( \vdash G < 0 \):
  \begin{align*}
    & \bar{x} + \bar{y} - \bar{z} = 0 \\
    \Rightarrow ~ & \bar{z} = \bar{x} + \bar{y} \\
    \Rightarrow ~ & \pr(Z) = \pr(X) + \pr(Y)
  \end{align*}

  Ricapitolando: \( Z = X + Y \Rightarrow \pr(Z) = \pr(X) + \pr(Y) \).
\end{proof}

Per i numeri aleatori illimitati\footnote{quelli per cui \( \inf I(X) \), \( \sup I(X) \) o entrambi non esistono finiti} può non esistere nessun valore finito corrispondente alla nostra valutazione di \( \pr(X) \).

\begin{proposition}[Calcolare la previsione]
  Supponiamo di avere un numero aleatorio $X$ dove \( I(X) = \{ X_1, \ldots, X_n \} \), una funzione \( \phi : \mathbb{R} \to \mathbb{R} \) e sia \( E_i \equiv (X = x_i) \), allora vale che:
  \[ \pr(\phi(X)) = \sum_{i=1}^{n} \phi(x_i) \pr(X = x_i) \]
\end{proposition}

\begin{proof}
  \begin{align*}
    \pr(\phi(X)) &= \pr(\phi(X) \cdot 1) \\
    &= \pr(\phi(X)(E_1 + \ldots + E_n)) \\
    &= \pr(\phi(X) E_1) + \ldots + \pr(\phi(X) E_n) \\
    &= \sum_{i=1}^{n} \pr(\phi(X) E_i) \\
    &= \sum_{i=1}^{n} \pr(\phi(x_i) E_i) \\ % TODO commentare questo passaggio
    &= \sum_{i=1}^{n} \phi(x_i) \pr(E_i) \\
    &= \sum_{i=1}^{n} \phi(x_i) \pr(X = x_i)
  \end{align*}
\end{proof}

\subsection{Probabilità}
\begin{definition}[Probabilità]\label{def:probabilita}
  Dato un evento \( E \), \( \pr(E) \) si chiama anche probabilità di \( E \).
\end{definition}

\begin{proposition}[Proprietà della probabilità]
  Dalla monotonia della previsione\footnote{vedi proposizione \ref{pro:monotonia_previsione}} e dalla definizione di evento\footnote{vedi definizione \ref{def:evento}} segue che:
  \marginnote{Se \( E \equiv 0 \), \( E \) è un evento impossibile. Quando \( E \equiv 1 \), \( E \) è un evento certo.}
  \begin{gather*}
    0 \le \pr(E) \le 1 \\
    E \equiv 0 \Rightarrow \pr(E) = 0 \\
    E \equiv 1 \Rightarrow \pr(E) = 1
  \end{gather*}
\end{proposition}

\begin{proposition}[Operazioni sulle probabilità]
  Dalla linearità della previsione\footnote{vedi proposizione \ref{pro:linearita_previsione}}:

  Somma:
  \[ \pr(E_1 + E_2) = \pr(E_1) + \pr(E_2) \]

  Somma logica:
  \[ \pr(E_1 \vee E_2) = \pr(E_1 + E_2 - E_1 E_2) \Rightarrow \pr(E_1 \vee E_2) \le \pr(E_1 + E_2) \]
\end{proposition}

Per la monotonia della previsione\footnote{vedi proposizione \ref{pro:monotonia_previsione}} si ha che:
\[ \vdash E_1 \vee E_2 \le E_1 + E_2 \Rightarrow \pr(E_1 \vee E_2) \le \pr(E_1 + E_2) \]

Per una partizione: % TODO queste non sono altre proprietà?
\[ \vdash E_1 + \ldots + E_n = 1 \Rightarrow \sum \pr(E_i) = 1 \]

\begin{proposition}[Calcolare la probabilità]
  Per ogni evento $E$ logicamente dipendente\footnote{ovvero \( \forall i, 1 \le i \le n, E_i \subset E \)} da una partizione \( E_1, \ldots, E_n \), la probabilità di $E$ sarà pari a:
  \[ \pr(E) = \frac{\#\{i|E_i \subset E\}}{n} \]
  che non è altro che la nota formula
  \[ \pr(E) = \frac{\text{\# casi favorevoli}}{\text{\# casi possibili}} \]
\end{proposition}

Dunque la probabilità che si verifichi un determinato evento di una partizione di eventi equiprobabili è
\[ \pr(E_i) = \frac{1}{n} \]

\subsection{Probabilità e previsione subordinata}
\begin{definition}[Previsione subordinata]
  La previsione subordinata\footnote{detta anche previsione condizionata} di un numero aleatorio $X$ rispetto ad un evento $E$, denotata con \( \pr(X|E) \), è la previsione di $X$ sapendo che $E$ si sia verificato.
\end{definition}

Tramite il \emph{metodo della scommessa} possiamo calcolare il guadagno:
\[ G = \lambda E (X - \bar{x}) \]
La definizione data è molto simile a quella della previsione, alla quale la moltiplicazione per l'evento $E$ dona le seguenti caratteristiche:
\begin{itemize}
\item Se $E$ si verifica allora anche $X$ può verificarsi, e il guadagno è lo stesso della previsione non subordinata.
\item Se $E$ non si verifica la scommessa è annullata e \( G = 0 \).
\end{itemize}

Inoltre \( I(X|E) \subset I(X) \).

\newthought{Valgono le stesse proprietà} della previsione:
\begin{definition}[Monotonia]\label{pro:monotonia_previsione_subordinata}
  \[ \inf I(X|H) \le \pr(X|H) \le \sup I(X|H) \]
\end{definition}

\begin{definition}[Linearità]\label{pro:linearita_previsione_subordinata}
  \[ \pr(X + Y|H) = \pr(X|H) + \pr(Y|H) \]
  \[ \pr(\alpha X|H) = \alpha \pr(X|H) \]
\end{definition}

che seguono allo stesso modo dal principio di coerenza.

\subsection{Probabilità composta}
\begin{theorem}[Probabilità composta]\label{thm:probabilita_composta}
  Dati due eventi $A$ e $B$, si ha che:
  \marginnote{In alcuni testi la probabilità composta si denota con \( \pr(A\cap B) \)}
  \[ \pr(AB) = \pr(A)\pr(B|A) = \pr(B)\pr(A|B) \]
\end{theorem}

\begin{proof}
  Poniamo \( z = \pr(AB) \), \( x = \pr(A) \) e \( y = \pr(B|A) \).

  Quindi, se supponiamo una scommessa su tutti i valori otteniamo:
  \begin{align*}
    G &= c_1 (A - x) + c_2 A (B - y) + c_3 (AB - z) \\
    &= c_1A - c_1 x + c_2 AB - c_2 Ay + c_3 AB - c_3 z \\
    &= A(c_1 + c_2 B - c_2 y + c_3 B) - c_1 x - c_3 z \\
    &= A(c_1 + (c_2 + c_3)B - c_2 y) - c_1 x - c_3 z
  \end{align*}
  A questo punto basta porre \( c_2 = - c_3 \) e \( c_1 = c_2 y \) per annullare la parte aleatoria:
  \begin{align*}
    G &= - c_1 x - c_3 z \\
    &= c_2 (z - xy)
  \end{align*}

  Per evitare di violare il principio di coerenza dobbiamo avere:
  \[ z = xy \Rightarrow \pr(AB) = \pr(A) \pr(B|A) \]
\end{proof}

Se gli eventi sono indipendenti si ha che \( \pr(B|A) = \pr(B) \) e analogamente \( \pr(A|B) = \pr(A) \). Proprietà da cui discende il seguente corollario:
\begin{corollary}[Probabilità composta di eventi indipendenti]
  Dati due eventi $A$ e $B$ logicamente indipendenti, si ha che:
  \[ \pr(AB) = \pr(A)\pr(B) \]
\end{corollary}

Il teorema della probabilità composta può essere generalizzato alla previsione:
\begin{corollary}[Previsione composta]
  Dati due numeri aleatori $X$ ed $Y$, si ha che:
  \[ \pr(XY) = \pr(Y)\pr(X|Y) = \pr(X)\pr(Y|X) \] % TODO sicuro?
\end{corollary}
La dimostrazione è analoga al teorema della probabilità composta\footnote{vedi teorema \ref{thm:probabilita_composta}}.

\subsection{Probabilità totale}
Il teorema della probabilità totale è utile per valutare la previsione conoscendo la previsione subordinata.

\begin{theorem}[Probabilità totale]\label{thm:probabilita_totale}
  Sia \( E_1, \ldots, E_n \) una partizione e $X$ un numero aleatorio. Vale che:
  \[ \pr(X) = \sum_{i = 1}^{n} \pr(X|E_i)\pr(E_i) \]
\end{theorem}

\begin{proof}
  \begin{align*}
    \pr(X) &= \pr(X \cdot 1) \\
    &= \pr(X(E_1 + \ldots + E_n)) \\
    &= \pr(XE_1 + \ldots + XE_n) \\
    &= \sum_{i=1}^{n} \pr(XE_i) \\
    &= \sum_{i=1}^{n} \pr(X|E_i)\pr(E_i)
  \end{align*}
\end{proof}

\subsection{Teorema di Bayes}
\begin{theorem}[Teorema di Bayes]
  Dati due eventi $A$ ed $B$, con \( \pr(A) > 0 \), si ha:
  \[ \pr(B|A) = \frac{\pr(A|B)\pr(B)}{\pr(A)} \]
\end{theorem}

\begin{proof}
  Dal teorema della probabilità composta\footnote{vedi teorema \ref{thm:probabilita_composta}} si ha che:
  \begin{gather*}
    \pr(AB) = \pr(A)\pr(B|A) \\
    \Rightarrow \pr(B|A) = \frac{\pr(AB)}{\pr(A)} = \frac{\pr(A|B)\pr(B)}{\pr(A)}
  \end{gather*}
\end{proof}

\subsection{Correlazione tra eventi}
\begin{definition}[Eventi correlati positivamente]
  \label{def:eventi_correlati_positivamente}
  Un evento $E$ si dice correlato positivamente con $H$ se
  \[ \pr(E|H) > \pr(E) \]
  oppure, se \( \pr(H) > 0 \) e \( \pr(E) > 0 \)
  \[ \pr(EH) > \pr(E)\pr(H) \]

  Il verificarsi di $H$ aumenta la valutazione della probabilità di $E$.
\end{definition}

\begin{definition}[Eventi correlati negativamente]
  \label{def:eventi_correlati_negativamente}
  Un evento $E$ si dice correlato negativamente con $H$ se
  \[ \pr(E|H) < \pr(E) \]
  oppure, se \( \pr(H) > 0 \) e \( \pr(E) > 0 \)
  \[ \pr(EH) < \pr(E)\pr(H) \]

  Il verificarsi di $H$ diminuisce la valutazione della probabilità di $E$.
\end{definition}

\begin{definition}[Eventi non correlati]
  \label{def:eventi_non_correlati}
  Un evento $E$ si dice non correlato con $H$ se
  \[ \pr(E|H) = \pr(E) \]

  Il verificarsi di $H$ non cambia la valutazione della probabilità di $E$ e viceversa.
\end{definition}

Se $E$ è correlato positivamente con $H$, si ha che \( \tilde{E} \) è correlato negativamente e viceversa. % TODO dimostrazione

\subsection{Indipendenza stocastica}
Se due eventi non sono correlati\footnote{vedi definizione \ref{def:eventi_non_correlati}} sono \emph{stocasticamente indipendenti}, mentre il contrario non è sempre vero.
\marginnote{Stocastico viene dal greco stochastikós ``congetturale'', e significa ``relativo al calcolo delle probabilità''.
  Talvolta l'indipendenza stocastica viene chiamata indipendenza statistica.}
Generalizzando possiamo dire che:
\begin{definition}[Indipendenza stocastica]
  $n$ eventi \( E_1, \ldots, E_n \) sono stocasticamente indipendenti se:
  \[ \pr(E_1 \cdots E_n) = \pr(E_1) \cdots \pr(E_n) \]
\end{definition}

La definizione è analoga per gli eventi subordinati:
\begin{definition}[Indipendenza stocastica per eventi subordinati]
  Sia \( \mathcal{H} = \{ H_1, \ldots, H_n \} \) una partizione; due eventi \( E_1, E_2 \) sono stocasticamente indipendenti subordinatamente alla partizione \( \mathcal{H} \) se \( \forall i \in \{1,\ldots,n\} \):
  \[ \pr(E_1 E_2 | H_i) = \pr(E_1 | H_i)\pr(E_2|H_i) \]
\end{definition}

\begin{definition}[Indipendenza stocastica attraverso i costituenti]
  $n$ eventi \( E_1, \ldots, E_n \) sono stocasticamente indipendenti se e solo se per ogni costituente:
  \( Q = E_1^* \cdots E_n^* \) di \( E_1, \ldots, E_n \) vale che:
  \[ \pr(Q) = \pr(E_1^*) \cdots \pr(E_n^*) \]
\end{definition}

\section{Varianza e covarianza}
\subsection{Covarianza}
\begin{definition}[Covarianza]\label{def:covarianza}
  Dati due numeri aleatori $X$ ed $Y$, si definisce covarianza di $X$ ed $Y$:
  \begin{align*}
    \cov(X,Y) &= \pr((X - \pr(X))(Y - \pr(Y))) \\
    &= \pr(XY - \pr(X)Y - \pr(Y)X + \pr(X)\pr(Y)) \\
    &= \pr(XY) - \pr(X)\pr(Y)
  \end{align*}
\end{definition}

\begin{proposition}[Correlazione tra numeri aleatori tramite covarianza]
  Dalle definizioni di eventi correlati e della covarianza abbiamo che due numeri aleatori $X$ e $Y$ si dicono:
  \begin{itemize}
  \item \emph{correlati positivamente}\footnote{vedi definizione \ref{def:eventi_correlati_positivamente}} se \( \cov(X,Y) > 0 \)
  \item \emph{correlati negativamente}\footnote{vedi definizione \ref{def:eventi_correlati_negativamente}} se \( \cov(X,Y) < 0 \)
  \item \emph{non correlati}\footnote{vedi definizione \ref{def:eventi_non_correlati}} se \( \cov(X,Y) = 0 \)
  \end{itemize}
\end{proposition}

\begin{proposition}[Bilinearità della covarianza]\label{def:bilinearita_della_covarianza}
  \[ \cov(X+Y,Z) = \cov(X,Z) + \cov(Y,Z) \]
\end{proposition}

\begin{proof}
  Utilizzando la definizione di covarianza\footnote{vedi definizione \ref{def:covarianza}} e la linearità della previsione\footnote{vedi proposizione \ref{pro:linearita_previsione}}:
  \begin{align*}
    \cov(X + Y,Z) &= \pr((X+Y)Z) - \pr(X+Y)\pr(Z) \\
    &= \pr(XZ + YZ) - (\pr(X)+\pr(Y))\pr(Z) \\
    &= \pr(XZ) + \pr(YZ) - \pr(X)\pr(Z) - \pr(Y)\pr(Z) \\
    &= \pr(XZ) - \pr(X)\pr(Z) + \pr(YZ) - \pr(Y)\pr(Z) \\
    &= \cov(X,Y) + \cov(Y,Z)
  \end{align*}
\end{proof}

\begin{proposition}[Trasformazione lineare della covarianza]
  \label{pro:trasformazione_lineare_covarianza}
  \[ \cov(aX + b, cY + d) = ac ~ \cov(X,Y) \]
\end{proposition}

\begin{proof}
  Utilizzando la definizione di covarianza\footnote{vedi definizione \ref{def:covarianza}} e la linearità della previsione\footnote{vedi proposizione \ref{pro:linearita_previsione}}:
  \begin{align*}
    \cov(aX + b, cY + d) &= \pr(((aX + b) - \pr(aX + b))((cX + d) - \pr(cY + d))) \\
    &= \pr((aX + b - a\pr(X) - b)(cX + d - c\pr(Y) - d)) \\
    &= \pr((aX - a\pr(X))(cY - c\pr(Y))) \\
    &= \pr(a(X - \pr(X)) c(Y - \pr(Y))) \\
    &= ac ~ \cov(X,Y)
  \end{align*}
\end{proof}

\subsection{Varianza}
\begin{definition}[Varianza]\label{def:varianza}
  \begin{align*}
    \var(X) &= \cov(X,X) \\
    &= \pr(X^2) - \pr(X)^2 \\
    &= \pr((X - \pr(X)^2)
  \end{align*}
\end{definition}
\marginnote{Da notare che la varianza è una quantità sempre positiva}
Se \( \var(X) = 0 \) allora tutta la probabilità è concentrata nella previsione \( \pr(X) \), ovvero \( X \equiv \pr(X) \).

\newthought{Data la varianza} si ottengono:
\begin{definition}[Scarto quadratico medio]\label{def:scarto_quadratico_medio}
  \[ \sqm(X) = \sqrt{\var(X)} \]
\end{definition}

\begin{definition}[Previsione quadratica]\label{def:previsione_quadratica}
  \[ P_Q(X) = \sqrt{\pr(X^2)} \]
\end{definition}

\begin{proposition}[Trasformazione lineare della varianza]
  \label{pro:trasformazione_lineare_varianza}
  \[ \var(aX + b) = a^2 \var(X) \]
\end{proposition}

\begin{proof}
  La proprietà della varianza segue dalla trasformazione lineare della covarianza\footnote{vedi proposizione \ref{pro:trasformazione_lineare_covarianza}} sostituendo \( (cY + d) \) con \( (aX + b) \).
\end{proof}

\section{Regressione Lineare} % TODO organizzare
\begin{definition}[Regressione lineare]
  Date due variabili aleatorie $X$ ed $Y$ la regressione lineare ci permette di calcolare la migliore approssimazione lineare del legame tra di esse.
  L'approssimazione lineare è tanto buona quanto la seguente quantità è piccola:
  \[ \pr((Y - aX - b)^2) \]
  dove \( Y = aX + b \). % TODO e allora a che serve?

  Per fare in modo che quella quantità sia più piccola possibile bisogna trovare la coppia \( (a,b) \in \mathbb{R}^2 \) che minimizza il risultato, ovvero:
  \begin{gather*}
    a = \frac{\cov(X,Y)}{\var(X)} \\
    b = 0
  \end{gather*}
\end{definition}

\begin{proof}
  Supponendo che \( \pr(X) = \pr(Y) = 0 \),
  \begin{align*}
    \pr((Y - aX - b)^2) &= \pr(Y^2 + a^2 X^2 + b^2 - 2aXY - 2bY + 2abX) \\
    &= \pr(Y^2) + a^2\pr(X^2) - 2a\pr(XY) - 2b\pr(Y) + 2ab\pr(X)
  \end{align*}
  Dalla definizione di varianza\footnote{vedi definizione \ref{def:varianza}} e covarianza\footnote{vedi definizione \ref{def:covarianza}} e dalle ipotesi abbiamo che \( \pr(Y)^2 = \var(Y) \) e \( \cov(X,Y) = \pr(XY) \).
  Dunque:
  \[ = \var(Y) + a^2 \var(X) + b^2 -2a \cov(X,Y) \] % TODO questa è complessa, semplificare

  Per minimizzare il risultato avremo sicuramente \( b = 0 \), dato che $b$ è soltanto una costante additiva, mentre per $a$ dobbiamo trovare i punti dove la derivata è nulla:
  \begin{gather*}
    \frac{d}{da} \var(Y) + a^2 \var(X) + b^2 -2a \cov(X,Y) = 2a\var(X) - 2 \cov(X,Y) \\
    \Rightarrow 2a\var(X) - 2 \cov(X,Y) = 0\\
    \Rightarrow a = \frac{\cov(X,Y)}{\var(X)}
  \end{gather*}

\end{proof}

\section{Disuguaglianze di Čebyšëv} % TODO organizzare
\begin{definition}[Prima disuguaglianza di Čebyšëv]\label{def:chebyshev1}
  Sia $X$ un numero aleatorio tale che \( P_Q(X) > 0 \), allora per ogni \( t > 0 \) vale che
  \[ \pr(|X| \ge t P_Q(X)) \le \frac{1}{t^2} \]
\end{definition}

\begin{proof}
  Sia l'evento \( E = (|X| \ge t P_Q(X)) \). Calcoliamo \( \pr(X^2) \) con il teorema delle probabilità totali\footnote{vedi teorema \ref{thm:probabilita_totale}}:
  \[ \pr(X^2) = \pr(X^2|E)\pr(E) + \pr(X^2|\tilde{E})\pr(\tilde{E}) \] % TODO manca qualcosa...
  Per la monotonia della previsione\footnote{vedi proposizione \ref{pro:monotonia_previsione}} si ha che \( \pr(X^2|\tilde{E}) \ge 0 \) in quanto \( X^2 \) è un numero aleatorio sempre positivo.
  Ne segue che:
  \[ \pr(X^2) \ge \pr(X^2|E)\pr(E) \]
  Quando si verifica  \( E \Rightarrow X^2 \ge (t P_Q(X))^2 \):
  \[ \pr(X^2) \ge t^2 P_Q(X)^2 \pr(E) \]
  Poiché\footnote{vedi definizione \ref{def:previsione_quadratica}} \( P_Q(X)^2 = \pr(X^2) \) si ottiene:
  \[ \pr(E) \le \frac{1}{t^2} \equiv \pr(|X| \ge t P_Q(X)) \le \frac{1}{t^2} \]
\end{proof}

\begin{definition}[Seconda disuguaglianza di Čebyšëv]\label{def:chebyshev2}
  Sia un numero aleatorio $X$ con \( \var(X) > 0 \). Posto \( m = \pr(X) \), \( \forall t > 0 \) si ha che:
  \[
  \pr(|X - m) \ge \sqm(X)t) \le \frac{1}{t^2}
  \]
\end{definition}

\begin{proof}
  Segue dalla prima disuguaglianza di Čebyšëv\footnote{vedi definizione \ref{def:chebyshev1}}, sostituendo ad $X$ il numero aleatorio \( Y = X - m \).
\end{proof}

\section{Legge debole dei grandi numeri}
La legge debole dei grandi numeri\footnote{nota anche come teorema di Bernoulli} fornisce una formalizzazione del fatto che la frequenza di un evento in un grande numero di prove ripetute indipendenti è vicino alla sua probabilità.

\begin{theorem}[Legge debole dei grandi numeri]\label{thm:teorema_bernoulli}
  Sia \( X_1,X_2, \ldots \) una successione di numeri aleatori a due a due non correlati, ovvero con \( \cov(X_i, X_j) ~ i \ne j \), equamente distribuiti ed aventi la stessa previsione \( \pr(X_i) = m \) e varianza \( \var(X_i) = \var \).
  % TODO varianza o sqm? ed è giusta quella eq? se si che senso ha?
  Definiamo una nuova variabile:
  \[ X \equiv \frac{X_1 + \ldots + X_n}{n} \]
  Sarà:
  \[ \forall \lambda > 0 ~~ \lim_{n \rightarrow \infty} \pr(|X - m| \ge \lambda) = 0 \]
\end{theorem}

\begin{proof}
  Dato che \( n \rightarrow \infty \):
  \begin{align*}
    \pr(X) &= \pr \left( \frac{X_1 + \ldots + X_n}{n} \right) \\
    &= \frac{1}{n} (\pr(X_1) + \ldots + \pr(X_n)) \\
    &= \frac{n~m}{n}
  \end{align*}
  Inoltre,
  \begin{align*}
    \var(X) &= \var \left( \frac{X_1 + \ldots + X_n}{n} \right) \\
    &= \var \left( \frac{X_1}{n} \right) + \ldots + \var \left( \frac{X_n}{n} \right) \\
    &= \frac{\var}{n^2} + \ldots + \frac{\var}{n^2} \\
    &= \frac{\var}{n}
  \end{align*}
  Dunque, per la disuguaglianza di Čebyšëv\footnote{vedi definizione \ref{def:chebyshev1}}, per ogni \( \lambda > 0 \),
  \[ \pr(|X - m| \ge \lambda) \le \frac{\var(X)}{\lambda^2} = \frac{\var}{n \lambda^2} \]
  Siccome \( n \rightarrow \infty \), segue che
  \[ \lim_{n \rightarrow \infty} \pr(|X - m| \ge \lambda) = 0 \]
\end{proof}

\section{Coefficiente di correlazione}
Il coefficiente di correlazione è la misura di quanto siano vicini i dati alla curva di approssimazione.

\begin{definition}[Coefficiente di correlazione]
  Dati due numeri aleatori $X$ ed $Y$, si definisce coefficiente di correlazione di $X,Y$
  \[ \cc(X,Y) = \frac{\cov(X,Y)}{\sqm(X)\sqm(Y)} \]
\end{definition}

\begin{proposition}[Trasformazione lineare del coefficiente di correlazione]
  \[ \cc(aX + b, cY + d) = \sgn(ac) \cc(X,Y) \]
\end{proposition}

\begin{proof}
  Utilizzando le proprietà della covarianza\footnote{vedi proposizione \ref{pro:trasformazione_lineare_covarianza}} e la definizione della funzione segno
  \footnote{\( \sgn(x) = \frac{x}{|x|} \mbox{~con~} x \ne 0 \)}:
  \begin{align*}
    \cc(aX + b, cY + d) &= \frac{\cov(aX + b, cY + d)}{\sqrt{\var(aX + b) \var(cY + d)}} \\
    &= \frac{ac ~ \cov(X,Y)}{|ac| \sqrt{\var(X) \var(Y)}} \\
    &= \sgn(ac) \cc(X,Y)
  \end{align*}
\end{proof}

\begin{definition}[Numero aleatorio standardizzato]
  \[ X^* = \frac{X - \pr(X)}{\sqm(X)} \]
\end{definition}
I numeri aleatori standardizzati hanno due caratteristiche interessanti: \( \pr(X^*) = \pr(Y^*) = 0 \) e \( \var(X^*) = \var(Y^*) = 1 \)

\begin{proposition}[Monotonia del coefficiente di correlazione]
  \[ -1 \le \cc(X,Y) \le 1 \]
\end{proposition}

\begin{proof}
  Dati due numeri aleatori standardizzati $X^*$ ed $Y^*$ si ha che:
  \begin{align*}
    \cov(X^*, Y^*) &= \pr(X^* Y^*) \\ % TODO espandere questo passaggio
    &= \frac{\pr((X - \pr(X))(Y - \pr(Y)))}{\sqm(X)\sqm(Y)} \\
    &= \cc(X,Y)
  \end{align*}
  Calcolando la varianza di $X^*$ ed $Y^*$ si ottiene:
  \begin{align*}
    \var(X^* + Y^*) &= \var(X^*) + \var(Y^*) + 2 \cov(X^*, Y^*) \\
    &= 1 + 1 + 2 \cov(X^*, Y^*) \\
    &= 2 + 2\cc(X,Y) \ge 0
  \end{align*}
  in quanto la varianza di un numero aleatorio è una quantità sempre positiva. Analogamente dalla varianza di \( X^* - Y^* \) segue che:
  \[ \var(X^* - Y^*) = 2 - 2\cc(X,Y) \ge 0 \]

  Quindi vale che:
  \[ -1 \le \cc(X,Y) \le 1 \]
\end{proof}

%!TEX root = appunti.tex
%!TEX encoding = UTF-8 Unicode
\chapter{Distribuzioni discrete}
\begin{definition}[Distribuzione discreta]\label{def:distribuzione_discreta}
  Una distribuzione statistica le cui variabili possono avere solo valori discreti.

  Dunque, un numero aleatorio $X$ si dice con distribuzione discreta se la cardinalità di $I(X)$ è finita o numerabile.
\end{definition}
\begin{definition}[Distribuzione di probabilità]\label{def:distribuzione_di_probabilita}
  La funzione che descrive la probabilità che un certo valore si verifichi.
  \marginnote{La distribuzione di probabilità viene anche detta ``funzione di densità di probabilità''.}

  Se consideriamo $X$ una variabile aleatoria con distribuzione discreta, la sua distribuzione di probabilità sarà data da
  \[ \pr(X = x_i) = p(x_i) \text{ con } x_i \in I(X) \]
  dove
  \[ \sum_{x_i \in I(X)} \pr(X = x_i) = 1 \]
\end{definition}

\begin{definition}[Schema di Bernoulli]\label{def:schema_bernoulli}
  Una successione \( (E_i)_{i \in \mathbb{N}} \) di eventi stocasticamente indipendenti ed equiprobabili, ovvero tali per cui vale che
  \[ \pr(E_i) = p, ~ \forall i \in \mathbb{N} \]
\end{definition}

\section{Distribuzione binomiale} % (fold)
\begin{figure*}
  \includegraphics{distribuzione_binomiale}
  \caption{Distribuzione binomiale} 
\end{figure*}

\begin{definition}[Distribuzione binomiale]
  \label{def:distribuzione_binomiale}
  La distribuzione discreta della probabilità di ottenere $k$ successi su $n$ prove di Bernoulli.
  \footnote{ovvero dove il risultato è positivo con probabilità $p$ e negativo con probabilità \( (1-p) \), vedi definizione \ref{def:schema_bernoulli}}
\end{definition}

Dato uno schema di Bernoulli \( (E_i)_{i \in \mathbb{N}} \) con \( \pr(E_i) = p \),
\[ S_n = (E_1 + \ldots + E_n) \]
è il numero di successi su $n$ prove dove, ovviamente,
\[ I(S_n) = \{0,1, \ldots, n\} \]

La distribuzione di probabilità dunque è, tramite i costituenti:
\[ \pr(S_n = k) = \sum_{Q \subset (S_n = k)} \pr(Q) \]
ovvero dobbiamo sommare tutte le probabilità dei costituenti del primo tipo dell'evento \( (S_n = k) \), come, ad esempio
\[ Q = E_1 \cdots E_k \tilde{E}_{k+1} \cdots \tilde{E}_n \]
che rappresenta l'evento in cui i $k$ successi si sono ottenuti con le prime $k$ prove.
Analogamente ogni altro costituente di \( (S_n = k) \) conterrà $k$ eventi che si sono verificati e \( n-k \) che non si sono verificati.
Siccome tutti gli eventi sono indipendenti ed equamente distribuiti, ogni costituente $Q$ ha la stessa probabilità, pari a
\[ \pr(Q) = p^k (1-p)^{n-k} \]

Dunque per avere \( \pr(S_n = k) \), basta contare i costituenti.
Essi sono uguali al numero di modi di scegliere i $k$ eventi che si verificano sugli $n$ totali. Si ottiene quindi
\[
  \pr(S_n = k) = \binom{n}{k} p^k (1-p)^{n-k} 
\]

Si dice dunque che \( S_n \) ha distribuzione binomiale \( \mathcal{B}(n, p) \) dove $n$ è il numero degli eventi e $p$ la probabilità di ognuno.

\newthought{Calcoliamo infine la previsione di \( S_n \)}:
\begin{align*}
  \pr(S_n) &= \pr(E_1 + \ldots + E_n) \\
  &= \sum_{i=1}^n \pr(E_i) \\
  &= np
\end{align*}

\newthought{e la sua varianza}:
\begin{align*}
  \var(S_n) &= \var(E_1 + \ldots + E_n) \\
  &= \sum_{i = 1}^n \var(E_i) \\
  &= \sum_{i = 1}^n \pr(E_i^2) - \pr(E_i)^2 \\
  &= \sum_{i = 1}^n \pr(E_i) - \pr(E_i)^2 \\
  &= \sum_{i = 1}^n p - p^2 \\
  &= \sum_{i = 1}^n p (1 - p) \\
  &= n p (1 - p)
\end{align*}

% TODO 'moda' della distribuzione

% section distribuzione_binomiale (end)

\section{Distribuzione ipergeometrica} % (fold)
\begin{figure*}
  \includegraphics{distribuzione_ipergeometrica}
  \caption{Distribuzione ipergeometrica} 
\end{figure*}

\begin{definition}[Distribuzione ipergeometrica]
  \label{def:distribuzione_ipergeometrica}
  La distribuzione discreta della probabilità di ottenere \( i \) successi su \( n \) campioni, date \( N \) possibilità totali su \( H \) considerate di successo.
\end{definition}
Consideriamo \( H \) modi per una ``buona'' selezione ed \( N - H \) per una ``cattiva'' selezione, su un totale di \( N \) possibilità.
Prendiamo \( n \) campioni e sia
\[ E_i = (\text{abbiamo una ``buona'' selezione all'}i\text{-esimo campione.}) \]
Sia \( X \) il numero aleatorio che conta il numero di buone selezioni:
\[
  X = \sum_{i = 1}^n E_i
\]

La probabilità di \( i \) selezioni di successo allora è
\begin{align*}
  \pr(X = i) &= \frac{[\text{\# modi per }i\text{ successi}][\text{\# modi per }n - i\text{ fallimenti}]}{[\text{\# casi possibili}]} \\
  &= \frac{\displaystyle \binom{H}{i} \binom{N-H}{n-i}}{\displaystyle \binom{N}{n}}
\end{align*}

Si dice dunque che \( X \) possiede distribuzione ipergeometrica.

\newthought{La probabilità dell'evento \( E_i \)} è data da:
\begin{align*}
  \pr(E_i) &= \frac{\text{\# casi favorevoli}}{\text{\# casi possibili}} \\
  &= \frac{H D_{n-1}^{N-1}}{D_n^N} \\
  &= \frac{H}{N}
\end{align*}
dove \( D_k^n \) è una disposizione semplice\footnote{vedi definizione \ref{def:disposizione_semplice}} di \( k \) elementi su \( n \) possibili.

\newthought{Dunque la previsione di \( X \) è}:
\begin{align*}
  \pr(X) &= \sum_{i=1}^n \pr(E_i) \\
  &= n \frac{N}{N}
\end{align*}

% section distribuzione_ipergeometrica (end)

\section{Distribuzione geometrica} % (fold)
\begin{figure*}
  \includegraphics{distribuzione_geometrica}
  \caption{Distribuzione geometrica} 
\end{figure*}

\begin{definition}[Distribuzione geometrica]\label{def:distribuzione_geometrica}
  La distribuzione discreta della probabilità di ottenere il primo successo all'istante $i$ in una serie di prove.
\end{definition}

Dato uno schema di Bernoulli \( E_i \) ed un numero aleatorio \( T \) che rappresenta l'istante del primo successo in una serie di prove, ovvero \( T = \inf\{ n | E_n = 1 \} \).
\[
  I(T) = \mathbb{N} \setminus \{ 0 \}
\]
\[
  (T = i) = \tilde{E}_1 \cdot \tilde{E}_{i-1} E_i
\]

La distribuzione discreta di probabilità è
\begin{align*}
  \pr(T = i) &= \pr(\tilde{E}_1 \cdot \tilde{E}_{i-1} E_i) \\
  &= \pr(\tilde{E}_1) \cdot \pr(\tilde{E}_{i-1}) \pr(E_i) \\
  &= (1-p)^{i-1} p  
\end{align*}

Si dice che \( T \) ha distribuzione geometrica di parametro \( p \). Utilizzando 
la somma della serie geometrica, si verifica che
\[
  \sum_{i=1}^{+ \infty} \pr(T = i) = 1
\]

\newthought{Possiamo dunque calcolare la previsione di \( T \)}:
\[
  \pr(T) = \frac{1}{p}
\]

% section distribuzione_geometrica (end)


\backmatter
% \bibliographystyle{plain}
% \bibliography{}
\end{document}
